\vfill
\textit{This chapter discusses the architecture of following modules:}
\begin{itemize}
    \item Transmitter.
    \begin{itemize}
        \item Chaos Generator.
        \item Modulator.
        \item Transmitter Finite State Machine.
    \end{itemize}
    \item Receiver.

\end{itemize}
\vfill


\newpage
In this chapter, we discuss the architecture of the modem, as well as the design choices
made. Figure. \ref{fig:full_mod} shows our proposed architecture to implement the modulator.
Figure. \ref{fig:dmod} shows the demodulator.



\begin{sidewaysfigure}
    \label{fig:full_mod}
    \caption{Transmitter Block Diagram}
    \includegraphics[width=\textwidth]{Top Level/tx.pdf}
\end{sidewaysfigure}

\begin{figure}[p]
    \label{fig:dmod}
    \caption{Demodulator Block Diagram}
    \includegraphics[width=\textwidth]{Top Level/demod.png}
\end{figure}

\section{Transmitter}
The modulator is split into three parts:
\begin{enumerate}
    \item Chaos generator.
    \item Modulator.
    \item Serializer.
\end{enumerate}

\subsection{Chaos Generator [RE-WRITE]}
The chaos is generated with a Logistic Map with $r = 4$. The value $4$ was chosen to simplify the multiplication by $r$ into a shift left by
$2$. The logistic map operates on Q2.6 numbers. From equation \ref{eqn:logistic_map} we can see that if $n_{i}$ is 8-bits in length,
then $n_{i+1}$ is 16-bits. This means that each cycle, the \textsc{Logistic Map} The multiplication is performed with the radix-4 Booth multiplication algorithm. As it
can be seen from the block digaram is shown in Fig.\ref{fig:booth}, all partial products are calculated in parallel meaning that each clock cycle, the module is capable
of generating $16$ bits of chaos.

However, calculating all four partial products in a single sycle each cycle consumes a lot of power. This can happen when the spreading factor is equal to $16$.
Also, 16-bits of chaos is not suffecient for fast data rates.
To solve this, we need to calculate a large number of chaos bits in a short amount of time.
The \textsc{Chaos Expander} takes 16 w y7awel l 256. kamel b2a

\subsection{Modulator}
The purpose of the \textsc{Modulator} is to modulate the chaos bits with the information bits.
The message is first loaded into the \textsc{Message PISO} and is then serially output. The output bit is xor-ed with
a delayed version of the chaos bit from \textsc{Chaos Generator} (delayed by $\beta$).

\begin{center}
    \begin{table}[h]
        \caption{Modulator Pin-Out}
        \vspace*{0.3cm}
    \begin{tabular}{|>{\centering\arraybackslash}m{0.2\textwidth}|>{\centering\arraybackslash}m{0.09\textwidth}|>{\centering\arraybackslash}m{0.1\textwidth}|>{\centering\arraybackslash}m{0.5\textwidth}|}
        \hline
        Name & Width & Direction & Description\\
        \hline
        \texttt{i\_clk} & 1 & Input & Positive edge clock.\\
        \texttt{i\_arst\_n} & 1 & Input & Active-low asynchronous reset.\\
        \texttt{i\_msg\_bit} & 1 & Input & The input message bit.\\
        \texttt{i\_chaos\_bit} & 1 & Input & The bit of chaos to be modulated.\\
        \texttt{i\_frame\_half} & 1 & Input & Asserted if (and only if) the index of the chip being sent is less than $\beta$.\\
        \texttt{i\_sf} & 2 & Input & The encoded spreading factor.\\
        \texttt{o\_modulated\_bit} & 1 & Output & The serial output of the \textsc{Modulator}.\\
        \hline
    \end{tabular}
    \end{table}
\end{center}

\subsection{Message Buffer}
The \textsc{Message Buffer} is a parallel-in serial-out shift register that stores the message to be sent.
\begin{center}
    \begin{table}[h]
        \caption{Serializer Pin-Out}
        \vspace*{0.3cm}
    \begin{tabular}{|>{\centering\arraybackslash}m{0.2\textwidth}|>{\centering\arraybackslash}m{0.09\textwidth}|>{\centering\arraybackslash}m{0.1\textwidth}|>{\centering\arraybackslash}m{0.5\textwidth}|}
        \hline
        Name & Width & Direction & Description\\
        \hline
        \texttt{i\_clk} & 1 & Input & Positive edge clock.\\
        \texttt{i\_arst\_n} & 1 & Input & Active-low asynchronous reset.\\
        \texttt{i\_data} & 32 & Input & Message to be sent\\
        \texttt{i\_load} & 1 & Input & Load the message from \texttt{i\_data} into the buffer.\\
        \texttt{i\_shift} & 1 & Input & Shift a bit from the message into \texttt{o\_bit}.\\
        \texttt{o\_bit} & 1 & Output & Serially output bit.\\
        \hline
    \end{tabular}
    \end{table}
\end{center}

\subsection{Chip/Bit Counter}
This counter keeps track of the index of the bit (and subsequently, the chip) that are begin sent.
Since the system allows changing the SF during runtime, the maximum value of the counter is 1024 for SF16.
The counter wraps around to zero based on the selected spreading factor according to Table \ref{tab:sf_wrap}
\begin{center}
    \begin{table}
        \label{tab:sf_wrap}
    \begin{tabular}{|c|c|}
        \hline
        SF &  Wrap-around value\\
        \hline
        SF2 & $2 \times 32 \times 2$ \\
        SF4 & $2 \times 32 \times 4$\\
        SF8 & $2 \times 32 \times 8$\\
        SF16 & $2 \times 32 \times 16$\\
        \hline
    \end{table}
    \end{tabular}
\end{center}

\section{Receiver}
At the heart of the receiver lies the MAC block, which by design, is a correlator. We don't care about the whole result of the accumulation,
but only its sign. The sign bit is the demodulated information bit. A SIPO buffer is used to store and output all 32 serially-demodulated bits.
w nktb h2a 7war enena bnst3ml 4 bits 34an 7war el nosie wel kalam el raye2 d2 (ya rb yb2a s7 bs)
\begin{center}
    \begin{table}[h]
        \caption{Demodulator Pin-Out}
        \vspace*{0.3cm}
    \begin{tabular}{|>{\centering\arraybackslash}m{0.2\textwidth}|>{\centering\arraybackslash}m{0.09\textwidth}|>{\centering\arraybackslash}m{0.1\textwidth}|>{\centering\arraybackslash}m{0.5\textwidth}|}
        \hline
        Name & Width & Direction & Description\\
        \hline
        \texttt{i\_clk} & 1 & Input & Positive edge clock.\\
        \texttt{i\_arst\_n} & 1 & Input & Active-low asynchronous reset.\\
        \texttt{i\_rx} & Q4.0 & Input & Recevied signal.\\
        \texttt{i\_recv} & 1 & Input & Start Receiving.\\
        \texttt{o\_valid} & 1 & Output & Asserted for one cycle after all message bits have been demodulated.\\
        \texttt{o\_msg} & 32 & Output & The demodulated message.\\
        \hline
    \end{tabular}
    \end{table}
\end{center}