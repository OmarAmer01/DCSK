\vfill
\textit{This chapter discusses the following:}
\begin{itemize}
    \item Introduction to Differential Chaos Shift-Keying.
    \item Logistic Map Overview.
    \item Proposed Architecture.
\end{itemize}
\vfill
\newpage

\section{Introduction to DCSK}
Differential chaos shift-keying (\textbf{DCSK}) is a modulation scheme based on using \textit{chaos} to modulate a signal
instead of a sinusoidal carrier. The reference signal (chaos) is sent over the channel, then after half a $T_s$ the reference signal is
modulated with the infromation signal and sent to the RX side. The composition of a DCSK frame
can be seen in the Equation. \ref{eqn:dcsk_frame}.

\begin{equation} \label{eqn:dcsk_frame}
    e_k =
        \begin{cases}
            x_k & \text{for } 1 < k < \beta\\
            s_i x_{k-\beta}  & \text{for } \beta < k \le 2\beta
        \end{cases}
\end{equation}
Where $\beta$ is an integer and $k$ is the chip index.\\

$e_k$ is sent over an AWGN channel, the signal at the RX side is given by:
\begin{equation}
    r_{sig} = e_k + n_k
\end{equation}

The received signal is correlated with a delayed version of itself (delayed by half a symbol's duartion) and then
the sign of the result is the demodulated infromation bit.

\section{An Overview of the Logistic Map}
Many methods exists to generate a chaotic signal. The Logistic Map is the most simple to understand and is the least complex to implement.
The Logistic Map is given by the following equation:
\begin{equation} \label{eqn:logistic_map}
    n_{i+1} = r \times n_i(1-n_i)
\end{equation}

Most values of $r$ beyond $\approx 3.56995$ exhibit chaotic behavior. We can see in Figure \ref{fig:logistic_map}
The effect of different values of $r$ on $n_{i+1}$.\\
\textcolor{red}{insert diagram here}\\
